%
% THE BEER-WARE LICENSE (Rev. 42):
% Ronny Bergmann <bergmann@mathematik.uni-kl.de> wrote this file. As long as you
% retain this notice you can do whatever you want with this stuff. If we meet
% some day, and you think this stuff is worth it, you can buy me a beer or
% coffee in return.
%
% This file is just to get started - You need the corresponding Logo
%


%\documentclass[german,10pt,xcolor=colortbl,compress
%,draft
%]{beamer}
\documentclass[german,10pt,xcolor=colortbl,compress,handout]{beamer}
\usepackage[macce]{inputenc}
\usepackage[OT1]{fontenc}
\usepackage{calc}
\usepackage[ngerman]{babel} % Neue Rechtschreibung
\usepackage{amsmath,amsthm,amssymb,euscript} % AMS-LaTeX  
\usepackage{enumerate,graphicx}
\usepackage{listings}
\usepackage{tikz}

% Load Theme
\usetheme[noptsans,navigation=true, FB=Mathematik, frametotal=true]{TUKL}
%
\setbeamertemplate{navigation symbols}{}
\title{Real root isolation}
\subtitle{Fachpraktikum}
\date[]{\today}
\author[]{Dominik Bendle, Clara Petroll}
\institute[]{AG Algebra, Geometrie und Computeralgebra \\ Supervisor: Janko B�hm}
%Setze ein Logo auf der Titelseite unten rechts
\renewcommand{\theSecondLogo}{}

\begin{document}
	\maketitle	
	\begin{frame}{Inhalt}
		\tableofcontents
	\end{frame}
	\section{Introduction}
	\begin{frame}{Problem}
		\begin{itemize}
		\item Given: zero-dimensional radical ideal  $J \subseteq \mathbb{R}[x_1, \hdots, x_n]$ and a box  $B=I_1 \times \cdots \times I_n$.
		\pause
		\item Find boxes $B_i \subseteq B$, such that any box contains exactly one element of $V(J)$.
		\end{itemize}
	\pause
	
	Why does this task make sense?
	\pause	
	\begin{lemma}
	Let $J \subseteq \mathbb{R}[x_1, \hdots, x_n]$ be an ideal. Then the following are equivalent: 
	\begin{itemize}
		\item[a)] $J$ is zero-dimensional.
		\item[b)] $V(J)$is finite.
	\end{itemize}
	\end{lemma}
	
	\end {frame}
	
	\begin{frame}{Process}
		\begin{itemize}
		\item interval arithmetic and exclusion to find out if there is no solution in a box
		\pause
		\item multivariate Interval Newton step to get uniqueness of a solution in a box
		\pause 
		\item implementation in a Singular-Library
		\end{itemize}	
	\end{frame}
	
	\section{The algorithm}
	\begin{frame}{Interval arithmetic}
		\begin{definition} 
		Let  $\textbf{R} \subseteq \mathbb{R}^2$ be the set of all real intervals $[a,b], a < b$.
		An interval extension of a map $g: \mathbb{R}^n \to \mathbb{R}$ is a map $\textbf{g}: \textbf{R}^n \to \textbf{R}$ such that $g(x) \in \textbf{g}(\textbf{x}) $ for all $x \in \mathbb{R}$.
		\end{definition}
		
		Therewith we can implement a simple exclusion test.  \bigskip
		
		\emph{Problem:} How do we get uniqueness of a solution in a box? 
		
	\end{frame}
	
	\begin{frame}{Interval Newton Step}
		\begin{itemize}
		\pause
		\item Idea: We want to have a test $T$, which for a given map $f: \mathbb{R}^n \to \mathbb{R}^n$ and a given box $B_i$ returns: 
		 	\begin{itemize}
			\item $T(f,B_i)=-1$, if there is no $x \in B_i$ such that $f(x)=0$.
			\item $T(f,B_i)=1$,  if there is a unique $x \in B_i$ such that $f(x)=0$.
			\item $T(f,B_i)=0$, else.
			\end{itemize}
		\end{itemize}
		\pause
		To get this, we compute the interval Newton step of $f$ and a box $\textbf{x}$:
		$$ N(f, \textbf{x})= \hat{x} - \textbf{f'}(\textbf{x})^{-1}f(\hat{x})$$
		(where $\hat{x} \in \textbf{x}$ arbitrary point, $\textbf{f'}$ the interval extension of the Jacobian matrix of $f$)
	\end{frame}
	
	% Vllt beispiel bei dem matrix invertiert wird??
	\begin{frame}{Interval Newton step}
		\begin{theorem}
		Let $\textbf{x}$ be a box in $\mathbb{R}^n$ and $f:\mathbb{R}^n\to \mathbb{R}^n$. Then it holds:
		\begin{itemize}
		\item[a)] Every solution in $\textbf{x}$ is also in $N(f,\textbf{x})$. \\
		\pause
		\item[b)] If $N(f,\textbf{x}) \subseteq int(\textbf{x})$, then there is a unique solution in $\textbf{x}$, i.e. $T(f,\textbf{x})=1$.
		\end{itemize}
		\end{theorem}
		\pause
		
		\textbf{Implementation:} 
		\\
		Bisection until for every of the arising boxes one of the following conditions is satisfied:
		\pause
		\begin{itemize}
		\item in the box is exactly one solution.
		\item \glqq size\grqq{} of the box is under a certain bound.
		\end{itemize}
		
	\end{frame}
	
	
	\begin{frame}{Interval Newton Step}{Application to our problem}
		\begin{theorem}
		Let $I$ be a zero-dimensional radical ideal of $K[x_1, \hdots, x_n]$, where $K$ is a field. Then $I$ is generated by $n$ elements.
		\end{theorem}
		
		\begin{itemize}
		\item zero-dimensional radical ideal $I \subseteq \mathbb{R}[x_1, \hdots, x_n]$ is therefore given by a set $\{f_1, \hdots f_n\}, f_i \in \mathbb{R}[x_1, \hdots, x_n]; i=1,\hdots,n$ of generators.
		\pause
		\item consider the function  $f:=\left( \begin{array}{c} f_1\\ \vdots \\ f_n \end{array} \right)$, $f:\mathbb{R}^n\to \mathbb{R}$  \\
		\pause	
		%\item Das haben wir weil... (computational methods in commutative Algebra and Algebraic geometry -> let I zeroddim radical then I generated by n elements)
		%\pause
		\end{itemize}
		$\implies$ We can apply the interval Newton step to our ideal.
	\end{frame}
			
	\section{Problems and improvements}
	\begin{frame}{Bad run time}
		\begin{itemize}
		\item interval arithmetic implemented as a \glqq newstruct\grqq{} in Singular
		\item inverting of an interval matrix difficult
		\end{itemize}
	\bigskip
	\textbf{Our improvements:}
	\pause
		\begin{itemize}
		\item outsourcing of the interval arithmetic in C++ (dynamical modules in Singular)
		\item integrate this part as .so
		\item inverting of the interval matrices with Gaussian elimination
		\end{itemize}
	\end{frame}
	
	\begin{frame}{Big fractions}
		\begin{itemize}
			\pause
			\item intersecting boxes $\rightarrow$ sometimes really big fractions $\rightarrow$ slow
			\pause
			\item Idea: "round" the boxes 
			\pause
			\item Realization: increase the size of the box a little bit, to get nicer fractions.
		\end{itemize}
		
		\begin{center}
		\begin{tikzpicture}[scale=1.75]
		\fill [color=yellow]  (1.25,1) rectangle (1.75,2);
		\draw[thick] (0,0) grid [step=1]     (2,2);
		\draw[thick,red] (1.25,0.75) -- (1.25,2.25)-- (1.75,2.25) -- (1.75, 0.75) -- (1.25,0.75);
		\draw [<-, very thick, blue] (1, 1.5) -- (1.25,1.5);
		\draw [->, very thick, blue] (1.75,1.5) -- (2,1.5);
		\end{tikzpicture}
		\end{center}
		\pause
		
		$\implies$ Thereby passable run time even for five variables.
	\end{frame}
	
	\begin{frame}{Bad starting boxes}
	\pause
	\begin{lemma}
	Let $I$ be an ideal of the polynomial ring $K[x_1, \hdots, x_n]$, where $K$ is a field. Then the following are equivalent:
	\begin{itemize}
	 	\item[a)] $I$ is zero-dimensional.
		\item[b)] $I \cap K[x_i] \neq \{0\}$ for $i=1,\hdots, n$.
	\end{itemize}
	\pause
	
	\end{lemma}
		\begin{itemize}
		\item Compute Gr�bner basis of the ideal with an elimination ordering.
		\pause
		\item $\implies$ First equation only depends of one variable.
		\pause
		\item Apply our exclusion test on this equation $\rightarrow$ we get better starting boxes for this variable.
		\item Do this for all variables (by permuting the variables).
		\item Take the cartesian product of all those boxes as starting boxes.
		\end{itemize}
	\end{frame}
	
	\begin{frame}{Boxes with unique solutions}
		\begin {itemize}
		\item So far: Input consisting of an ideal, starting box and a bound for the size of the boxes.
		%\item Wollen nur Boxen mit eindeutigen L�sungen.
		\\ $\rightsquigarrow$ \emph{Aim:} We don't want to have boxes wich are returned after termination just because they are too small.
		\end{itemize}
	\pause
	\textbf{Our improvements:}
	\begin {itemize}
	\pause
	\item If there is a zero on the plane, at which we want to divide the box for bisection  $\implies$ wouldn't get uniqueness of the solution with interval Newton step.
	\item Same problem if we have a zero on the boundary of the starting box.
	\pause
	\item Improve our algorithm \glqq splitBox\grqq{} by:
		\pause
		\begin{itemize}
		\item evaluate the ideal at the intersection plane
		\item Gr�bner basis test
		\end{itemize}
	\end{itemize}
	
	\end{frame}
	% warum k�nnen wir newton step anwenden?
	% nulldimensionales Radikalideal -> immer n erzeuge??, auf jeden fall finite vanishing locus
	
	% frame zu Unser Vorgehen 
	%Probleme: sehr langsam-> L�sung?
	%Codeausschnitte
	
	%Literatur
	
	\begin{frame}{Bibliography}
	\bibliographystyle{alphadin}
	\begin{thebibliography}{9}
	\bibitem{Moore} Cloud, Kearfott, Moore
	\textit{Introduction to Interval Analysis}.
	Society for Industrial and Applied Mathematics, 2009.
	
	
	\bibitem{Vasconcelos} Eisenbud, Grayson, Herzog, Stillman, Vasconcelos
	\textit{Computational Methods in Commutative Algebra and Algebraic Geometry}. Springer Verlag Berlin-Heidelberg, 3. edition 2004.
	
	\bibitem[A]{Sommese} Andrew J. Sommese and Charles W. Wampler
	\textit{The Numerical Solution of Systems of Polynomials}. [\textit{Arising in Engineering and Science}].
	World Scientific Publishing Co. Pte. Ltd., 2005.
	
	\end{thebibliography}
	
	\end{frame}
	
\end{document}